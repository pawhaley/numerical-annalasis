\documentclass[12pt]{article}

\usepackage{amssymb,amsmath,amsthm}
\usepackage[top=1in, bottom=1in, left=1.25in, right=1.25in]{geometry}
\usepackage{fancyhdr}
\usepackage{enumerate}
\usepackage[bw,framed,numbered]{mcode}
\usepackage{graphicx}

% Comment the following line to use TeX's default font of Computer Modern.
\usepackage{times,txfonts}

\newtheoremstyle{homework}% name of the style to be used
  {18pt}% measure of space to leave above the theorem. E.g.: 3pt
  {12pt}% measure of space to leave below the theorem. E.g.: 3pt
  {}% name of font to use in the body of the theorem
  {}% measure of space to indent
  {\bfseries}% name of head font
  {:}% punctuation between head and body
  {2ex}% space after theorem head; " " = normal interword space
  {}% Manually specify head
\theoremstyle{homework} 

% Set up an Exercise environment and a Solution label.
\newtheorem*{exercisecore}{Exercise \@currentlabel}
\newenvironment{exercise}[1]
{\def\@currentlabel{#1}\exercisecore}
{\endexercisecore}

\newcommand{\localhead}[1]{\par\smallskip\noindent\textbf{#1}\nobreak\\}%
\newcommand\solution{\localhead{Solution:}}

%%%%%%%%%%%%%%%%%%%%%%%%%%%%%%%%%%%%%%%%%%%%%%%%%%%%%%%%%%%%%%%%%%%%%%%%
%
% Stuff for getting the name/document date/title across the header
\makeatletter
\RequirePackage{fancyhdr}
\pagestyle{fancy}
\fancyfoot[C]{\ifnum \value{page} > 1\relax\thepage\fi}
\fancyhead[L]{\ifx\@doclabel\@empty\else\@doclabel\fi}
\fancyhead[C]{\ifx\@docdate\@empty\else\@docdate\fi}
\fancyhead[R]{\ifx\@docauthor\@empty\else\@docauthor\fi}
\headheight 15pt

\def\doclabel#1{\gdef\@doclabel{#1}}
\doclabel{Use {\tt\textbackslash doclabel\{MY LABEL\}}.}
\def\docdate#1{\gdef\@docdate{#1}}
\docdate{Use {\tt\textbackslash docdate\{MY DATE\}}.}
\def\docauthor#1{\gdef\@docauthor{#1}}
\docauthor{Use {\tt\textbackslash docauthor\{MY NAME\}}.}
\makeatother

% Shortcuts for blackboard bold number sets (reals, integers, etc.)
\newcommand{\Reals}{\ensuremath{\mathbb R}}
\newcommand{\Nats}{\ensuremath{\mathbb N}}
\newcommand{\Ints}{\ensuremath{\mathbb Z}}
\newcommand{\Rats}{\ensuremath{\mathbb Q}}
\newcommand{\Cplx}{\ensuremath{\mathbb C}}
%% Some equivalents that some people may prefer.
\let\RR\Reals
\let\NN\Nats
\let\II\Ints
\let\CC\Cplx

%%%%%%%%%%%%%%%%%%%%%%%%%%%%%%%%%%%%%%%%%%%%%%%%%%%%%%%%%%%%%%%%%%%%%%%%%%%%%%%%%%%%%%%
%%%%%%%%%%%%%%%%%%%%%%%%%%%%%%%%%%%%%%%%%%%%%%%%%%%%%%%%%%%%%%%%%%%%%%%%%%%%%%%%%%%%%%%
% 
% The main document start here.

% The following commands set up the material that appears in the header.

%%%%%%%%%%%%%%%%%%%%%%%%%%%%%%%%%%%%%%%%%%%%%%%%%%%%%%%%%%%%%%%%%%%%%%%%%%%%%%%%%%%%%%%
%%%%%%%%%%%%%%%%%%%%%%%%%%%%%%%%%%%%%%%%%%%%%%%%%%%%%%%%%%%%%%%%%%%%%%%%%%%%%%%%%%%%%%%
% 
% The main document start here.

% The following commands set up the material that appears in the header.
\doclabel{Math 310: Homework 6}
\docauthor{Parker Whaley}
\docdate{September 9, 2016}

\newcommand{\vv}{\mathbf{v}}
\begin{document}
\begin{enumerate}[(1)]
\item
Start with an easy case:
$$A =
\begin{bmatrix}
a_{11} & a_{12}\\
a_{21} & a_{22}
\end{bmatrix}
$$
In this case, all we need to do is clear $a_{21}$. How many foating point operations are needed?  Be eficient! When you have an answer, please discuss it with me!\\\\
Since we are storing the values neededd to reduce $A\rightarrow U$ in $L$ there are no additional floating point operations needed to construct $L$.  To construct $L$ we need to eliminate $a_{21}$, to do this we subtract $k=a_{21}/a_{11}$.  We need only compute the value going in the $22$ spot since the value in the $21$ spot will be 0.  We will then subteact $k*a_{12}$ from $a_{22}$ to find the value at that spot.  There are 3 operations, one to compute $k$ and two to subtract $k*a_{12}$.
\item
Next easiest case:
$$A =\begin{bmatrix}
a_{11}& a_{12}& a_{13}\\
a_{21}& a_{22}& a_{23}\\
a_{31}& a_{32}& a_{33}
\end{bmatrix}$$\\\\
We first need to eliminate the first row terms, calculate $k_2=a_{21}/a_{11}$ and $k_3=a_{31}/a_{11}$, two floating point ops.  Now set $a_{21}$ and $a_{31}$ to zero.  Subtract $k_2$ copies of row 1 from row two, except for the fist columb, four floating point operations, and the same with $k_3$ and the third row another four operations.  Now we have a two by two to reduce, wich we know takes 3 operations, for a total of 13 operations.
\item
What about a $4\times 4$ matrix?\\\\
In this case we will compute 3 k's in a similar fassion to the $3\times 3$, 3 floating point operations, then we will set the first columb except the top entry to zero's.  now subtract the first row times each multiplyer $k$ from the row beneath, ignoring the first row, we take $2*3*3$ operations.  Now we need to reduce a $3\times 3$ wich will take 13 operations, bringing our total to 34.
\item
What about a $n\times n$ matrix?\\\\
We will need $n-1$ operations to compute the k's, then $2(n-1)^2$ operations to eliminate the first entries.  After that we are left with a $n-1\times n-1$.  So assuming $f(n)$ is a function giving the number of operations to reduce a $n\times n$, we know that $f(n)=f(n-1)+n-1+2(n-1)^2$ and $f(1)=0$ since a $1\times 1$ is reduced.
$$f(n)=\sum_{i=1}^{n-1} i+2i^2=(n-1)(n)/2+(n-1)n(2n-1)/3$$.
\end{enumerate}

\end{document}