\documentclass[12pt]{article}

\usepackage{amssymb,amsmath,amsthm}
\usepackage[top=1in, bottom=1in, left=1.25in, right=1.25in]{geometry}
\usepackage{fancyhdr}
\usepackage{enumerate}
\usepackage[bw,framed,numbered]{mcode}
\usepackage{graphicx}

% Comment the following line to use TeX's default font of Computer Modern.
\usepackage{times,txfonts}

\newtheoremstyle{homework}% name of the style to be used
  {18pt}% measure of space to leave above the theorem. E.g.: 3pt
  {12pt}% measure of space to leave below the theorem. E.g.: 3pt
  {}% name of font to use in the body of the theorem
  {}% measure of space to indent
  {\bfseries}% name of head font
  {:}% punctuation between head and body
  {2ex}% space after theorem head; " " = normal interword space
  {}% Manually specify head
\theoremstyle{homework} 

% Set up an Exercise environment and a Solution label.
\newtheorem*{exercisecore}{Exercise \@currentlabel}
\newenvironment{exercise}[1]
{\def\@currentlabel{#1}\exercisecore}
{\endexercisecore}

\newcommand{\localhead}[1]{\par\smallskip\noindent\textbf{#1}\nobreak\\}%
\newcommand\solution{\localhead{Solution:}}

%%%%%%%%%%%%%%%%%%%%%%%%%%%%%%%%%%%%%%%%%%%%%%%%%%%%%%%%%%%%%%%%%%%%%%%%
%
% Stuff for getting the name/document date/title across the header
\makeatletter
\RequirePackage{fancyhdr}
\pagestyle{fancy}
\fancyfoot[C]{\ifnum \value{page} > 1\relax\thepage\fi}
\fancyhead[L]{\ifx\@doclabel\@empty\else\@doclabel\fi}
\fancyhead[C]{\ifx\@docdate\@empty\else\@docdate\fi}
\fancyhead[R]{\ifx\@docauthor\@empty\else\@docauthor\fi}
\headheight 15pt

\def\doclabel#1{\gdef\@doclabel{#1}}
\doclabel{Use {\tt\textbackslash doclabel\{MY LABEL\}}.}
\def\docdate#1{\gdef\@docdate{#1}}
\docdate{Use {\tt\textbackslash docdate\{MY DATE\}}.}
\def\docauthor#1{\gdef\@docauthor{#1}}
\docauthor{Use {\tt\textbackslash docauthor\{MY NAME\}}.}
\makeatother

% Shortcuts for blackboard bold number sets (reals, integers, etc.)
\newcommand{\Reals}{\ensuremath{\mathbb R}}
\newcommand{\Nats}{\ensuremath{\mathbb N}}
\newcommand{\Ints}{\ensuremath{\mathbb Z}}
\newcommand{\Rats}{\ensuremath{\mathbb Q}}
\newcommand{\Cplx}{\ensuremath{\mathbb C}}
%% Some equivalents that some people may prefer.
\let\RR\Reals
\let\NN\Nats
\let\II\Ints
\let\CC\Cplx

%%%%%%%%%%%%%%%%%%%%%%%%%%%%%%%%%%%%%%%%%%%%%%%%%%%%%%%%%%%%%%%%%%%%%%%%%%%%%%%%%%%%%%%
%%%%%%%%%%%%%%%%%%%%%%%%%%%%%%%%%%%%%%%%%%%%%%%%%%%%%%%%%%%%%%%%%%%%%%%%%%%%%%%%%%%%%%%
% 
% The main document start here.

% The following commands set up the material that appears in the header.

%%%%%%%%%%%%%%%%%%%%%%%%%%%%%%%%%%%%%%%%%%%%%%%%%%%%%%%%%%%%%%%%%%%%%%%%%%%%%%%%%%%%%%%
%%%%%%%%%%%%%%%%%%%%%%%%%%%%%%%%%%%%%%%%%%%%%%%%%%%%%%%%%%%%%%%%%%%%%%%%%%%%%%%%%%%%%%%
% 
% The main document start here.

% The following commands set up the material that appears in the header.
\doclabel{Math 310: Homework 6}
\docauthor{Parker Whaley}
\docdate{September 9, 2016}

\newcommand{\vv}{\mathbf{v}}
\begin{document}
\begin{enumerate}[(1)]
\item
Start with an easy case:
$$A =
\begin{bmatrix}
a_{11} & a_{12}\\
a_{21} & a_{22}
\end{bmatrix}
$$
In this case, all we need to do is clear $a_{21}$. How many floating point operations are needed?  Be efficient! When you have an answer, please discuss it with me!\\\\
Since we are storing the values needed to reduce $A\rightarrow U$ in $L$ there are no additional floating point operations needed to construct $L$.  To construct $L$ we need to eliminate $a_{21}$, to do this we subtract $k=a_{21}/a_{11}$.  We need only compute the value going in the $22$ spot since the value in the $21$ spot will be 0.  We will then subtract $k*a_{12}$ from $a_{22}$ to find the value at that spot.  There are 3 operations, one to compute $k$ and two to subtract $k*a_{12}$.
\item
Next easiest case:
$$A =\begin{bmatrix}
a_{11}& a_{12}& a_{13}\\
a_{21}& a_{22}& a_{23}\\
a_{31}& a_{32}& a_{33}
\end{bmatrix}$$\\\\
We first need to eliminate the first row terms, calculate $k_2=a_{21}/a_{11}$ and $k_3=a_{31}/a_{11}$, two floating point ops.  Now set $a_{21}$ and $a_{31}$ to zero.  Subtract $k_2$ copies of row 1 from row two, except for the fist column, four floating point operations, and the same with $k_3$ and the third row another four operations.  Now we have a two by two to reduce, witch we know takes 3 operations, for a total of 13 operations.
\item
What about a $4\times 4$ matrix?\\\\
In this case we will compute 3 k's in a similar fashion to the $3\times 3$, 3 floating point operations, then we will set the first column except the top entry to zero's.  now subtract the first row times each multiplier $k$ from the row beneath, ignoring the first row, we take $2*3*3$ operations.  Now we need to reduce a $3\times 3$ witch will take 13 operations, bringing our total to 34.
\item
What about a $n\times n$ matrix?\\\\
We will need $n-1$ operations to compute the k's, then $2(n-1)^2$ operations to eliminate the first entries.  After that we are left with a $n-1\times n-1$.  So assuming $f(n)$ is a function giving the number of operations to reduce a $n\times n$, we know that $f(n)=f(n-1)+n-1+2(n-1)^2$ and $f(1)=0$ since a $1\times 1$ is reduced.
$$f(n)=\sum_{i=1}^{n-1} i+2i^2=(n-1)(n)/2+(n-1)n(2n-1)/3$$.
\end{enumerate}
\begin{exercise}

Problem 7.1
\end{exercise}
$$A=\begin{bmatrix}
4 & -1 & -1\\
-1 & 4 & -1\\
-1 & -1 & 4
\end{bmatrix}$$
$$U_1=\begin{bmatrix}
4 & -1 & -1\\
-1 & 4 & -1\\
-1 & -1 & 4
\end{bmatrix}
L_1=\begin{bmatrix}
1 & 0 & 0\\
0 & 1 & 0\\
0 & 0 & 1
\end{bmatrix}$$
$$U_2=\begin{bmatrix}
4 & -1 & -1\\
0 & 3.75 & -1.25\\
0 & -1.25 & 3.75
\end{bmatrix}
L_2=\begin{bmatrix}
1 & 0 & 0\\
-.25 & 1 & 0\\
-.25 & 0 & 1
\end{bmatrix}$$
$$U_3=\begin{bmatrix}
4 & -1 & -1\\
0 & 3.75 & -1.25\\
0 & 0 & 10/3
\end{bmatrix}
L_3=\begin{bmatrix}
1 & 0 & 0\\
-.25 & 1 & 0\\
-.25 & -1/3 & 1
\end{bmatrix}$$
Starting over:
$$A=\begin{bmatrix}
4 & -1 & -1\\
-1 & 4 & -1\\
-1 & -1 & 4
\end{bmatrix}$$
Assume there exists a $L$ lower triangular where $A=LL^T$.
$$L=\begin{bmatrix}
a & 0 & 0\\
b & c & 0\\
d & e & f
\end{bmatrix}$$
$$LL^T=\begin{bmatrix}
a & 0 & 0\\
b & c & 0\\
d & e & f
\end{bmatrix}
\cdot
\begin{bmatrix}
a & b & d\\
0 & c & e\\
0 & 0 & f
\end{bmatrix}=\begin{bmatrix}
a^2 & ba & da\\
ba & b^2+c^2 & bd+ce\\
da & bd+ce & d^2+e^2+f^2
\end{bmatrix}$$
Lets solve for what we can, $a=2$
$$\begin{bmatrix}
4 & 2b & 2d\\
2b & b^2+c^2 & bd+ce\\
2d & bd+ce & d^2+e^2+f^2
\end{bmatrix}$$
$b=-1/2$, $d=-1/2$\\
$$\begin{bmatrix}
4 & -1 & -1\\
-1 & 1/4+c^2 & 1/4+ce\\
-1 & 1/4+ce & 1/4+e^2+f^2
\end{bmatrix}$$
$1/4+c^2=4\rightarrow c=1.9365$
$$\begin{bmatrix}
4 & -1 & -1\\
-1 & 4 & 1/4+ce\\
-1 & 1/4+ce & 1/4+e^2+f^2
\end{bmatrix}$$
$1/4+ce=-1\rightarrow e=(-1-1/4)/c=-0.64550$
$$\begin{bmatrix}
4 & -1 & -1\\
-1 & 4 & -1\\
-1 & -1 & 1/4+e^2+f^2
\end{bmatrix}$$
$1/4+e^2+f^2=4\rightarrow f=\sqrt{4-1/4-e^2}=1.8257$
$$\begin{bmatrix}
4 & -1 & -1\\
-1 & 4 & -1\\
-1 & -1 & 4
\end{bmatrix}$$
We now know that $LL^T=A$.
$$L=\begin{bmatrix}
2 & 0 & 0\\
-1/2 & 1.9365 & 0\\
-1/2 & -0.64550 & 1.8257
\end{bmatrix}$$
\begin{exercise}

Problem 7.2
\end{exercise}
\lstinputlisting{../octave/usolve.m}
Just to test this code let's use a random $U$ and $y$ and see if the $x$ produced works.
\newpage
\lstinputlisting{../octave/q2.txt}
It definitely worked.
\begin{exercise}

Problem 7.4
\end{exercise}
$$Ax=b$$
$$PAx=Pb$$
$$LUx=Pb$$
$$Ux=c$$
$$Lc=Pb$$

$$\begin{bmatrix}
1&0&0\\
1/2&1&0\\
1/3&1/4&1
\end{bmatrix}
c=\begin{bmatrix}
-12\\
2\\
10
\end{bmatrix}$$

$$\begin{bmatrix}
1&0&0\\
0&1&0\\
0&1/4&1
\end{bmatrix}
c=\begin{bmatrix}
-12\\
8\\
14
\end{bmatrix}$$

$$\begin{bmatrix}
1&0&0\\
0&1&0\\
0&0&1
\end{bmatrix}
c=\begin{bmatrix}
-12\\
8\\
12
\end{bmatrix}$$

$$
c=\begin{bmatrix}
-12\\
8\\
12
\end{bmatrix}$$

$$
Ux=\begin{bmatrix}
-12\\
8\\
12
\end{bmatrix}$$

$$
\begin{bmatrix}
2&3&1\\
0&1&2\\
0&0&2
\end{bmatrix}
x=\begin{bmatrix}
-12\\
8\\
12
\end{bmatrix}$$

$$
\begin{bmatrix}
2&3&0\\
0&1&0\\
0&0&2
\end{bmatrix}
x=\begin{bmatrix}
-18\\
-4\\
12
\end{bmatrix}$$

$$
\begin{bmatrix}
2&0&0\\
0&1&0\\
0&0&2
\end{bmatrix}
x=\begin{bmatrix}
-6\\
-4\\
12
\end{bmatrix}$$

$$
\begin{bmatrix}
1&0&0\\
0&1&0\\
0&0&1
\end{bmatrix}
x=\begin{bmatrix}
-3\\
-4\\
6
\end{bmatrix}$$

$$x=\begin{bmatrix}
-3\\
-4\\
6
\end{bmatrix}$$
\begin{exercise}

Problem 7.6
\end{exercise}
\begin{enumerate}[(a)]
\item
We simply add each entry to the corresponding entry in the other vector, $n$ additions.
\item
Well the product of two $n$ vectors is $n$ multiplications and $n-1$ additions.  To multiply a $m\times n$ by a $n$ vector we need to do $m$ products of two $n$ vectors, $m*n$ multiplications and $m*(n-1)$ additions.
\item
$$Ux=y$$
We need to do substitution, first calculate the lowest $x$ value,one division, then substitute it back in, $n-1$ multiplications and $n-1$ subtractions.  Do this for each row:\\
number of divisions $=\sum_{i=1}^{n}1=n$\\
number of multiplications $=\sum_{i=1}^{n}i-1=n(n+1)/2-n=n^2/2-n/2$\\
number of subtractions $=\sum_{i=1}^{n}i-1=n(n+1)/2-n=n^2/2-n/2$\\
\end{enumerate}
\begin{exercise}

Problem 6.2
\end{exercise}
\begin{enumerate}[(a)]
\item
Absolute condition number $C(x)=|f'(x)|=|1/3x^{-2/3}|$.\\
Relative condition number $K(x)=|C(x)x/f(x)|=1/3$.\\
\item
In a absolute sense $f$ is well conditioned away from zero.  In a relative sense it is well conditioned everywhere.
\item
$|f(\hat{x})-f(x)|\approx C(x)|\hat{x}-x|=1/3*10^{34/3}*9*10^{-16}=3*10^{-14/3}$
\end{enumerate}
\begin{exercise}

Problem 6.4
\end{exercise}
Here I show the error and values for $\tan(x)$ for the $x_j$ in the text:\\
\lstinputlisting{../octave/q6_4.txt}
$K(x)=|f'(x)x/f(x)|=|\frac{xcos(x)}{\cos^2(x)sin(x)}|=|\frac{x}{\cos(x)sin(x)}|$ If $x=x_j$ we then get $K(x)=2|x_j|$.  If we Examine $x_j=\pi/4+2\pi*10^{j}=\pi(1/4+2*10^{j})$, it is clear if we have errors in $\pi$ on the order of $10^{-16}$ we will get relative errors in $x_j$ on the order of $\frac{10^{-16}(1/4+2*10^{j})}{\pi(1/4+2*10^{j})}=\frac{10^{-16}}{\pi}$.  We would expect our error in $f(x_j)$ to be $|f(x_j)-f(\hat{x_j})|=f(x_j)*2|x_j|*\frac{10^{-16}}{\pi}=2|x_j|*\frac{10^{-16}}{\pi}$.  Now we can ask if this accurately models the error we see:
\lstinputlisting{../octave/q6_4b.txt}
It looks like this model for error works well until we get to high values.  This difference at high values can be explained by the nature of $\tan$.  Basically once your error in input is bigger than $\pi$, the repeat factor for $\tan$, your outputs are no longer based on your input at all, they are essentially the same as the output of $\tan$ evaluated at a random input, witch is far more likely to fall close to zero than far away from zero.  This explains why we didn't see our error match the predicted error for large $i$ values.
\end{document}