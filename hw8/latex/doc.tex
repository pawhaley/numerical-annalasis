\documentclass[12pt]{article}

\usepackage{amssymb,amsmath,amsthm}
\usepackage[top=1in, bottom=1in, left=1.25in, right=1.25in]{geometry}
\usepackage{fancyhdr}
\usepackage{enumerate}
\usepackage[bw,framed,numbered]{mcode}
\usepackage{graphicx}

% Comment the following line to use TeX's default font of Computer Modern.
\usepackage{times,txfonts}

\newtheoremstyle{homework}% name of the style to be used
  {18pt}% measure of space to leave above the theorem. E.g.: 3pt
  {12pt}% measure of space to leave below the theorem. E.g.: 3pt
  {}% name of font to use in the body of the theorem
  {}% measure of space to indent
  {\bfseries}% name of head font
  {:}% punctuation between head and body
  {2ex}% space after theorem head; " " = normal interword space
  {}% Manually specify head
\theoremstyle{homework} 

% Set up an Exercise environment and a Solution label.
\newtheorem*{exercisecore}{Exercise \@currentlabel}
\newenvironment{exercise}[1]
{\def\@currentlabel{#1}\exercisecore}
{\endexercisecore}

\newcommand{\localhead}[1]{\par\smallskip\noindent\textbf{#1}\nobreak\\}%
\newcommand\solution{\localhead{Solution:}}

%%%%%%%%%%%%%%%%%%%%%%%%%%%%%%%%%%%%%%%%%%%%%%%%%%%%%%%%%%%%%%%%%%%%%%%%
%
% Stuff for getting the name/document date/title across the header
\makeatletter
\RequirePackage{fancyhdr}
\pagestyle{fancy}
\fancyfoot[C]{\ifnum \value{page} > 1\relax\thepage\fi}
\fancyhead[L]{\ifx\@doclabel\@empty\else\@doclabel\fi}
\fancyhead[C]{\ifx\@docdate\@empty\else\@docdate\fi}
\fancyhead[R]{\ifx\@docauthor\@empty\else\@docauthor\fi}
\headheight 15pt

\def\doclabel#1{\gdef\@doclabel{#1}}
\doclabel{Use {\tt\textbackslash doclabel\{MY LABEL\}}.}
\def\docdate#1{\gdef\@docdate{#1}}
\docdate{Use {\tt\textbackslash docdate\{MY DATE\}}.}
\def\docauthor#1{\gdef\@docauthor{#1}}
\docauthor{Use {\tt\textbackslash docauthor\{MY NAME\}}.}
\makeatother

% Shortcuts for blackboard bold number sets (reals, integers, etc.)
\newcommand{\Reals}{\ensuremath{\mathbb R}}
\newcommand{\Nats}{\ensuremath{\mathbb N}}
\newcommand{\Ints}{\ensuremath{\mathbb Z}}
\newcommand{\Rats}{\ensuremath{\mathbb Q}}
\newcommand{\Cplx}{\ensuremath{\mathbb C}}
%% Some equivalents that some people may prefer.
\let\RR\Reals
\let\NN\Nats
\let\II\Ints
\let\CC\Cplx

%%%%%%%%%%%%%%%%%%%%%%%%%%%%%%%%%%%%%%%%%%%%%%%%%%%%%%%%%%%%%%%%%%%%%%%%%%%%%%%%%%%%%%%
%%%%%%%%%%%%%%%%%%%%%%%%%%%%%%%%%%%%%%%%%%%%%%%%%%%%%%%%%%%%%%%%%%%%%%%%%%%%%%%%%%%%%%%
% 
% The main document start here.

% The following commands set up the material that appears in the header.

%%%%%%%%%%%%%%%%%%%%%%%%%%%%%%%%%%%%%%%%%%%%%%%%%%%%%%%%%%%%%%%%%%%%%%%%%%%%%%%%%%%%%%%
%%%%%%%%%%%%%%%%%%%%%%%%%%%%%%%%%%%%%%%%%%%%%%%%%%%%%%%%%%%%%%%%%%%%%%%%%%%%%%%%%%%%%%%
% 
% The main document start here.

% The following commands set up the material that appears in the header.
\doclabel{Math 310: Homework 6}
\docauthor{Parker Whaley}
\docdate{September 9, 2016}

\newcommand{\vv}{\mathbf{v}}
\begin{document}
\begin{exercise}

7.8
\end{exercise}
One norm $=$ sum of abselute terms $=15$.  
Two norm $=$ root of the sum of squares $=\sqrt{77}$.  
Infinity norm $=$ largest abselute term $=6$.
\begin{exercise}

7.9
\end{exercise}
one norm $=$ largest of the sum of abselute terms in a columb $=14$.
infinity norm $=$ largest of the sum of abselute terms in a row $=15$.\\
Using octave $A^{-1}=\begin{bmatrix}
-4&3\\3.5&-2.5
\end{bmatrix}$.  Wich has a one norm of $7.5$ and a infinity norm of $7$.  So its one norm condition number is $105$ and its infinity condition number is $105$.

\begin{exercise}

7.10
\end{exercise}
Let $v$ be a $n$-vector.  WLoG let $v_1$ be the maximum abselute entry in $v$, and thus $|v_1|=||v||_\infty$.
\begin{enumerate}[(a)]
\item
Note that for all $i$, $v_i^2\leq v_1^2$.  Thus $v_1^2\leq v_1^2 + \sum_{i=2}^n v_i^2\leq n*v_1^2$ or $\sqrt{v_1^2}\leq \sqrt{v_1^2 + \sum_{i=2}^n v_i^2}\leq \sqrt{n*v_1^2}$ or $||v||_\infty\leq ||v||_2 \leq \sqrt{n}||v||_\infty$.
\item
Note that $(|a|+|b|)^2=a^2+b^2+2|ab|\geq a^2+b^2$  by trivial induction we can conclude $\sum |v_i|\geq \sqrt{\sum v_i^2}$ or $||v||_1\geq ||v||_2$.
\item
Note that $v_i\leq v_1$.  Thus $\sum v_i\leq nv_1$ or $||v||_1\leq n||v||_\infty$.
\end{enumerate}
\begin{enumerate}[(a)]
\item
There will never be equality for any non zero vector $v$ living in $n>1$ space.  We can see this by simply considering the proposed equality, $||v||_\infty= ||v||_2 = \sqrt{n}||v||_\infty$, or in other words $||v||_\infty= \sqrt{n}||v||_\infty$ so $||v||_\infty=0$, witch only occurs for $v=\vec{0}$.
\item
The vector $v=\begin{pmatrix}
1\\0
\end{pmatrix}$ has the property that $||v||_1= ||v||_2= 1$.
\item
The vector $v=\begin{pmatrix}
1\\1
\end{pmatrix}$ has the property that $||v||_1= 2||v||_\infty=2$.
\end{enumerate}
\begin{exercise}

4
\end{exercise}
\lstinputlisting{../octave/d1.txt}
\lstinputlisting{../octave/PPsolve.m}
\lstinputlisting{../octave/myplu.m}
\lstinputlisting{../octave/lsolve.m}
\lstinputlisting{../octave/usolve.m}
\begin{exercise}

5
\end{exercise}
TBA???
\end{document}