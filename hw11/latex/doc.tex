\documentclass[12pt]{article}

\usepackage{amssymb,amsmath,amsthm}
\usepackage[top=1in, bottom=1in, left=1.25in, right=1.25in]{geometry}
\usepackage{fancyhdr}
\usepackage{enumerate}
\usepackage[bw,framed,numbered]{mcode}
\usepackage{graphicx}

% Comment the following line to use TeX's default font of Computer Modern.
\usepackage{times,txfonts}

\newtheoremstyle{homework}% name of the style to be used
  {18pt}% measure of space to leave above the theorem. E.g.: 3pt
  {12pt}% measure of space to leave below the theorem. E.g.: 3pt
  {}% name of font to use in the body of the theorem
  {}% measure of space to indent
  {\bfseries}% name of head font
  {:}% punctuation between head and body
  {2ex}% space after theorem head; " " = normal interword space
  {}% Manually specify head
\theoremstyle{homework} 

% Set up an Exercise environment and a Solution label.
\newtheorem*{exercisecore}{Exercise \@currentlabel}
\newenvironment{exercise}[1]
{\def\@currentlabel{#1}\exercisecore}
{\endexercisecore}

\newcommand{\localhead}[1]{\par\smallskip\noindent\textbf{#1}\nobreak\\}%
\newcommand\solution{\localhead{Solution:}}

%%%%%%%%%%%%%%%%%%%%%%%%%%%%%%%%%%%%%%%%%%%%%%%%%%%%%%%%%%%%%%%%%%%%%%%%
%
% Stuff for getting the name/document date/title across the header
\makeatletter
\RequirePackage{fancyhdr}
\pagestyle{fancy}
\fancyfoot[C]{\ifnum \value{page} > 1\relax\thepage\fi}
\fancyhead[L]{\ifx\@doclabel\@empty\else\@doclabel\fi}
\fancyhead[C]{\ifx\@docdate\@empty\else\@docdate\fi}
\fancyhead[R]{\ifx\@docauthor\@empty\else\@docauthor\fi}
\headheight 15pt

\def\doclabel#1{\gdef\@doclabel{#1}}
\doclabel{Use {\tt\textbackslash doclabel\{MY LABEL\}}.}
\def\docdate#1{\gdef\@docdate{#1}}
\docdate{Use {\tt\textbackslash docdate\{MY DATE\}}.}
\def\docauthor#1{\gdef\@docauthor{#1}}
\docauthor{Use {\tt\textbackslash docauthor\{MY NAME\}}.}
\makeatother

% Shortcuts for blackboard bold number sets (reals, integers, etc.)
\newcommand{\Reals}{\ensuremath{\mathbb R}}
\newcommand{\Nats}{\ensuremath{\mathbb N}}
\newcommand{\Ints}{\ensuremath{\mathbb Z}}
\newcommand{\Rats}{\ensuremath{\mathbb Q}}
\newcommand{\Cplx}{\ensuremath{\mathbb C}}
%% Some equivalents that some people may prefer.
\let\RR\Reals
\let\NN\Nats
\let\II\Ints
\let\CC\Cplx

%%%%%%%%%%%%%%%%%%%%%%%%%%%%%%%%%%%%%%%%%%%%%%%%%%%%%%%%%%%%%%%%%%%%%%%%%%%%%%%%%%%%%%%
%%%%%%%%%%%%%%%%%%%%%%%%%%%%%%%%%%%%%%%%%%%%%%%%%%%%%%%%%%%%%%%%%%%%%%%%%%%%%%%%%%%%%%%
% 
% The main document start here.

% The following commands set up the material that appears in the header.

%%%%%%%%%%%%%%%%%%%%%%%%%%%%%%%%%%%%%%%%%%%%%%%%%%%%%%%%%%%%%%%%%%%%%%%%%%%%%%%%%%%%%%%
%%%%%%%%%%%%%%%%%%%%%%%%%%%%%%%%%%%%%%%%%%%%%%%%%%%%%%%%%%%%%%%%%%%%%%%%%%%%%%%%%%%%%%%
% 
% The main document start here.

% The following commands set up the material that appears in the header.
\doclabel{Math 310: Homework 10}
\docauthor{Parker Whaley}
\docdate{November 14, 2016}

\newcommand{\vv}{\mathbf{v}}
\begin{document}
\begin{exercise}
1
Problem 10.5
\end{exercise}
We are defining $$<a|b>=\int_{-1}^1 ab(1-x^2)^{-1/2}dx$$
Find the orthonormal basis using $[1,x,x^2]$.
Starting with $1$ lets normalize note $<1|1>=\int_{-1}^1 (1-x^2)^{-1/2}dx=\sin^{-1}(x)|_{-1}^1=\pi$.  Thus $q_0=1/\sqrt{\pi}$.  Note $<x|1/\sqrt{\pi}>=1/\sqrt{\pi}\int_{-1}^1 x(1-x^2)^{-1/2}dx=-1/\sqrt{\pi}\sqrt{1-x^2}|_{-1}^1=0$ and $<x|x>=\int_{-1}^1 x^2(1-x^2)^{-1/2}dx=1/2 (-x \sqrt{1 - x^2} + \sin^{-1}(x))=\pi/2$, thus $q_1=x\sqrt{2/pi}$.  Note $1/\sqrt{\pi}<x^2|1/\sqrt{\pi}>=1/\pi\int_{-1}^1 x^2(1-x^2)^{-1/2}dx=1/2$, and $<x^2|x\sqrt{2/pi}>=\sqrt{2/pi}\int_{-1}^1 x^3(1-x^2)^{-1/2}dx=0$ and $<x^2-1/2|x^2-1/2>=\pi/8$ (last integral via WRA), thus $q_2=x^2\sqrt{8/\pi}-\sqrt{2/\pi}$.\\
Note that $q_1=\alpha\cos(0)$, $q_1=\alpha\cos(\arccos(x))$, $q_2=\alpha (2x^2+1)=\cos(2\arccos(x))$ where the various $\alpha$s are constants so our basis is the same as the one given in the book.
\begin{exercise}
2
Problem 10.7
\end{exercise}
\lstinputlisting{../octave/traprule.m}
\lstinputlisting{../octave/simpson.m}
Note that realInt came from the described quad evaluation.
\lstinputlisting{../octave/d1.txt}

\begin{exercise}
3
Continuing with the theme that some sample points are better than others, recall that
polynomial interpolation with high-order polynomials is prone to making large oscillation
errors, but that this can be minimized using Chebyshev polynomials, which are the
Lagrange polynomials associated with the sample points
$x_j = cos(\pi + (\pi j/n))$, $j = 0, \cdots , n$
on the interval $[-1,1]$. Clenshaw-Curtis integration is integration using polynomial interpolation
at these sample points.
Use the MATLAB polyfit function to perform polynomial interpolation at these sample
points for $n = 4, 6,10$ and then use the resulting polynomials to approximate
$$\int^1_{-1}x \sin(x) dx$$
Compare your approximations to the exact answer (which you should compute by hand).
Integration by parts!
\end{exercise}
integration by parts (tabular method)\\
\begin{tabular}{c c c}
$+$ & $x$ & $\sin(x)$\\
$-$ & $1$ & $-\cos(x)$\\
$+$ & $0$ & $-\sin(x)$\\
\end{tabular}\\
Thus
$$\int^1_{-1}x \sin(x) dx=2(\sin(1)-\cos(1))$$
so our error for the three would be
\lstinputlisting{../octave/cheb.m}
\lstinputlisting{../octave/d2.txt}
\begin{exercise}
4
Recall that 5 point Gauss-Legendre integration uses sample points$[-\beta,-\alpha,0,\alpha,\beta]$ where
$$\alpha=1/3\sqrt{5-2\sqrt{10/7}}$$
$$\beta=1/3\sqrt{5+2\sqrt{10/7}}$$
Write a code that performs composite Gauss-Legendre integration with these sample
points. Your code should have the signature\\
q=glquad(f,a,b,N)\\
were f is the function to integrate, a and b are the endpoints of integration, and N is the
number of subintervals. Your code should perform Gauss-Legendre integration on each
subinterval and add them up. Then apply your function to compute
$$\int^1_{-1}x \sin(x) dx$$
using $N = 1, 2, 4,10$. Compare the results of Gauss-Legendre integration to the results
you saw using Clenshaw-Curtis integration.
\end{exercise}
\lstinputlisting{../octave/glquad.m}
\lstinputlisting{../octave/glerr.m}
\lstinputlisting{../octave/d3.txt}
Let's see where we can compare the two.  When GL is using one interval it uses $5$ points, so lets compare it to the $4$ point lenshaw-Curtis, both are on the $10^{-4}$ error.  When GL is using two intervals it uses $10$ points, so lets compare it to the $8$ point lenshaw-Curtis, GL has error $10^{-6}$, while lenshaw-Curtis has error $10^{-10}$.  Clearly lenshaw-Curtis does better using the same number of sample points for this function.
\begin{exercise}
5
Problem 9.1
\end{exercise}
\lstinputlisting{../octave/secderiv.m}
\lstinputlisting{../octave/d4.txt}
If we notice that we are deviding by $h^2$ we can easily see why the error starts to build.  Quite quickly $h^2$ becomes larger than machine precision an so it makes seance why our error gets big after that, to illustrate I have also given the table containing $h^2$ values.
\begin{exercise}
6
Problem 9.5
\end{exercise}
Recall via Taylor expansion $f(x+a)=f(x)+f'(x)a+f''(\xi)a^2/2$.  So we note that $f(x+h)=f(x)+f'(x)h+f''(\xi_1)h^2/2$ where $\xi_1\in(x,x+h)$.  Note that $f(x+2h)=f(x)+f'(x)2h+f''(\xi_2)2h^2$ where $\xi_2\in(x,x+2h)$.  Thus
$$\frac{1}{2h}[-3f(x)+4f(x+h)-f(x+2h)]=$$
$$\frac{1}{2h}[-3f(x)+4(f(x)+f'(x)h+f''(\xi_1)h^2/2)-(f(x)+f'(x)2h+f''(\xi_2)2h^2)]=$$
$$\frac{1}{2h}[4f'(x)h+4f''(\xi_1)h^2/2-f'(x)2h-f''(\xi_2)2h^2]=$$
$$f'(x)+\frac{1}{2h}[4f''(\xi_1)h^2/2-f''(\xi_2)2h^2]=$$
$$f'(x)+h[f''(\xi_1)-f''(\xi_2)]$$
Thus the error in this approximation is $h[f''(\xi_1)-f''(\xi_2)]$, this is interesting, if we have a function with a stable second derivative the error in this estimation will be tiny.











\end{document}