\documentclass[12pt]{article}

\usepackage{amssymb,amsmath,amsthm}
\usepackage[top=1in, bottom=1in, left=1.25in, right=1.25in]{geometry}
\usepackage{fancyhdr}
\usepackage{enumerate}
\usepackage[bw,framed,numbered]{mcode}
\usepackage{graphicx}

% Comment the following line to use TeX's default font of Computer Modern.
\usepackage{times,txfonts}

\newtheoremstyle{homework}% name of the style to be used
  {18pt}% measure of space to leave above the theorem. E.g.: 3pt
  {12pt}% measure of space to leave below the theorem. E.g.: 3pt
  {}% name of font to use in the body of the theorem
  {}% measure of space to indent
  {\bfseries}% name of head font
  {:}% punctuation between head and body
  {2ex}% space after theorem head; " " = normal interword space
  {}% Manually specify head
\theoremstyle{homework} 

% Set up an Exercise environment and a Solution label.
\newtheorem*{exercisecore}{Exercise \@currentlabel}
\newenvironment{exercise}[1]
{\def\@currentlabel{#1}\exercisecore}
{\endexercisecore}

\newcommand{\localhead}[1]{\par\smallskip\noindent\textbf{#1}\nobreak\\}%
\newcommand\solution{\localhead{Solution:}}

%%%%%%%%%%%%%%%%%%%%%%%%%%%%%%%%%%%%%%%%%%%%%%%%%%%%%%%%%%%%%%%%%%%%%%%%
%
% Stuff for getting the name/document date/title across the header
\makeatletter
\RequirePackage{fancyhdr}
\pagestyle{fancy}
\fancyfoot[C]{\ifnum \value{page} > 1\relax\thepage\fi}
\fancyhead[L]{\ifx\@doclabel\@empty\else\@doclabel\fi}
\fancyhead[C]{\ifx\@docdate\@empty\else\@docdate\fi}
\fancyhead[R]{\ifx\@docauthor\@empty\else\@docauthor\fi}
\headheight 15pt

\def\doclabel#1{\gdef\@doclabel{#1}}
\doclabel{Use {\tt\textbackslash doclabel\{MY LABEL\}}.}
\def\docdate#1{\gdef\@docdate{#1}}
\docdate{Use {\tt\textbackslash docdate\{MY DATE\}}.}
\def\docauthor#1{\gdef\@docauthor{#1}}
\docauthor{Use {\tt\textbackslash docauthor\{MY NAME\}}.}
\makeatother

% Shortcuts for blackboard bold number sets (reals, integers, etc.)
\newcommand{\Reals}{\ensuremath{\mathbb R}}
\newcommand{\Nats}{\ensuremath{\mathbb N}}
\newcommand{\Ints}{\ensuremath{\mathbb Z}}
\newcommand{\Rats}{\ensuremath{\mathbb Q}}
\newcommand{\Cplx}{\ensuremath{\mathbb C}}
%% Some equivalents that some people may prefer.
\let\RR\Reals
\let\NN\Nats
\let\II\Ints
\let\CC\Cplx

%%%%%%%%%%%%%%%%%%%%%%%%%%%%%%%%%%%%%%%%%%%%%%%%%%%%%%%%%%%%%%%%%%%%%%%%%%%%%%%%%%%%%%%
%%%%%%%%%%%%%%%%%%%%%%%%%%%%%%%%%%%%%%%%%%%%%%%%%%%%%%%%%%%%%%%%%%%%%%%%%%%%%%%%%%%%%%%
% 
% The main document start here.

% The following commands set up the material that appears in the header.

%%%%%%%%%%%%%%%%%%%%%%%%%%%%%%%%%%%%%%%%%%%%%%%%%%%%%%%%%%%%%%%%%%%%%%%%%%%%%%%%%%%%%%%
%%%%%%%%%%%%%%%%%%%%%%%%%%%%%%%%%%%%%%%%%%%%%%%%%%%%%%%%%%%%%%%%%%%%%%%%%%%%%%%%%%%%%%%
% 
% The main document start here.

% The following commands set up the material that appears in the header.
\doclabel{Math 310: Homework 10}
\docauthor{Parker Whaley}
\docdate{November 14, 2016}

\newcommand{\vv}{\mathbf{v}}
\begin{document}
\begin{exercise}

Problem 8.12
\end{exercise}
Show that $$s(x)=\begin{cases}
1+x-x^3 & 0\leq x<1\\
1-2(x-1)-3(x-1)^2+4(x-1)^3 & 1\leq x<2\\
4(x-2)+9(x-2)^2-3(x-2)^3 & 2\leq x<3
\end{cases}$$
is a cubic spline through $(0,1),(1,1),(2,0),(3,10)$.\\\\
For notation $1^+$ means a little more than 1 and $1^-$ is a little less than 1, for dealing with the peace wise function.\\
First lets make sure the values link up.\\
$s(x=0)=1+x-x^3=1$ good on $(0,1)$\\
$s(x=1^-)=1+x-x^3=1$ and $s(x=1^+)=1-2(x-1)-3(x-1)^2+4(x-1)^3=1$ good on $(1,1)$\\
$s(x=2^-)=1-2(x-1)-3(x-1)^2+4(x-1)^3=0$ and $s(x=2^+)=4(x-2)+9(x-2)^2-3(x-2)^3=0$ good on $(2,0)$\\
$s(x=3)=4(x-2)+9(x-2)^2-3(x-2)^3=10$ good on $(3,10)$\\
Now check that the derivatives mach.\\
$s'(x=1^-)=1-3x^2=-2$ and $s'(x=1^+)=-2-6(x-1)+12(x-1)^2=-2$ good.\\
$s'(x=2^-)=-2-6(x-1)+12(x-1)^2=4$ and $s'(x=2^+)=4+18(x-2)-9(x-2)^2=4$ good.\\
Now check that the second derivatives mach.\\
$s''(x=1^-)=-6x=-6$ and $s''(x=1^+)=-6+24(x-1)=-6$ good.\\
$s''(x=2^-)=-6+24(x-1)=18$ and $s''(x=2^+)=18-18(x-2)=18$ good.\\
Therefore it is a cubic spline.
Note that $s''(x=0)=-6x=0$ and $s''(x=3)=18-18(x-2)=0$ thus $s$ is a natural cubic spline.
\begin{exercise}

Problem 8.13
\end{exercise}
For what constants is
$$s(x)=\begin{cases}
ax^2+b(x-1)^3 & x\in(-\infty,1]\\
cx^2+d & x\in[1,2]\\
ex^2+f(x-2)^3 & x\in[1,2]\\
\end{cases}$$
a cubic spline?\\
Note
$$s'(x)=\begin{cases}
2ax+3b(x-1)^2 & x\in(-\infty,1]\\
2cx & x\in[1,2]\\
2ex+3f(x-2)^2 & x\in[1,2]\\
\end{cases}$$
$$s''(x)=\begin{cases}
2a+6b(x-1) & x\in(-\infty,1]\\
2c & x\in[1,2]\\
2e+6f(x-2) & x\in[1,2]\\
\end{cases}$$
We require that up to the second derivative link so our constraints are:\\
values\\
$a=c+d$\\
$4c+d=4e$\\
derivatives\\
$2a=2c$\\
$4c=4e$\\
second derivatives\\
$2a=2c$\\
$2c=2e$\\
These equations hold true if and only if $d=0$, $a=c=e$.  Note that this allows us 3 degrees of freedom in choosing our constants, $b$, $f$ and one of $a=c=e$.
\begin{exercise}

Find $n$ so that degree $n$ polynomial interpolation of $f (x) = \cos(3x)$, using equally spaced
points on $[0, 2]$, gives a maximum approximation error $|f (x) - p(x)|$ which is
less than $10^{-6}$ on $[0, 2]$.
\end{exercise}
Note that by inspection our polynomial will be at least order $2$ since a line is a terrible approximation for $\cos$ and thus $0$ order and $1$ order won't work.\\
The max error on a polynomial interpolation would be
$$e=|\frac{f^{(n+1)}(\xi)}{(n+1)!}*\Pi_{i=0}^{n}(\eta-x_i)|$$
where $\xi$ and $\eta$ are in $[0,2]$.  Note that for equally spaced points $\Pi_{i=0}^{n}|\eta-x_i|$ achieves its maximum with $\eta\in(x_0,x_1)$ and again at $\eta\in(x_{n-1},x_n)$.  The proof for this is trivial just think about the symmetry and what happens when you shift left or right by $x_1-x_0$ (one of those two shifts is guaranteed to give you a increase).  Let $\eta_0$ be the $\eta_0\in(x_0,x_1)$ that makes $\Pi_{i=0}^{n}|\eta-x_i|$ a maximum.  Thus 
$\Pi_{i=0}^{n}|\eta-x_i|\leq
\Pi_{i=0}^{n}|\eta_0-x_i|=
(\eta_0-x_0)\Pi_{i=1}^{n}(x_i-\eta_0)\leq 
(x_1-x_0)\Pi_{i=1}^{n}(x_i-x_0)=
\frac{2}{n}\Pi_{i=1}^{n}(i\frac{2}{n})=
(\frac{2}{n})^{n+1}\Pi_{i=1}^{n}i=
(\frac{2}{n})^{n+1}n!$.  Note that $f^{(n+1)}(x)=3^{n+1}S(3x)$ where $S(x)$ is a $\pm\sin(x)$ or $\pm\cos(x)$.  Noting that $|S(x)|\leq 1$ conclude $|f^{(n+1)}(\xi)|\leq 3^{n+1}$.  Now we can conclude 
$$e=|\frac{f^{(n+1)}(\xi)}{(n+1)!}*\Pi_{i=0}^{n}(\eta-x_i)|\leq
\frac{3^{n+1}}{(n+1)!}*(\frac{2}{n})^{n+1}n!=
\frac{6^{n+1}}{n^{n+1}(n+1)}
$$
So we can guarantee a $e<10^{-6}$ if $\frac{6^{n+1}}{n^{n+1}(n+1)}<10^{-6}$.
\lstinputlisting{../octave/d1.txt}
From this test we can see that the error will be guaranteed to be under $10^{-6}$ for a polynomial of degree 14.  My suspicion would be that a polynomial of order 13 will work since it is guaranteed to have a error less than $1.4216e-06$, but based on this math I can only guarantee that a polynomial of degree 14 will work.\\
We can now run some code to see what will work.
\lstinputlisting{../octave/errfind.m}
Witch ends up telling us that a polynomial of degree 12 will work.
\begin{exercise}

At the bottom of page 198 is an inequality that describes the error from the piece wise linear
interpolate $l(x)$ for $f (x)$ on $[a, b]$. Suppose we have equally spaced points $a =
x_0 < x_1 < \dots < x_n = b$ with spacing $h = xi - x_{i-1}$
.  then:
$$|f (x) - l(x)| \leq
\frac{Mh^2}{
8}$$
for all $x \in [a, b]$. In this inequality we are assuming $f
''(x)$ exists and is bounded by the
number $M$, so that $|f
''(x)| \leq M$ for all $x \in [a, b]$. Use this inequality to find $n$ so that
$|f (x) - l(x)| \leq 10^{-6}$
for $x \in [0, 2]$ if $f (x) = \cos(3x)$.
\end{exercise}
Well $M=9$ works since $|f'' (x)| = 9|\cos(3x)|\leq 9$.  Solving we get $sqrt(10^{-6}*8/9)^{-1}*2=2121.3$ so $n=2122$ will work.

\begin{exercise}

10.1
\end{exercise}
Derive the Newton-Cotes formula between $[0,1]$ for a 3 regions.\\
Polynomial interpolation allows us to say 
$$f(x) \approx \sum_{i=0}^{3} f(x_i)*\prod_{j\neq i}\frac{x-x_j}{x_i-x_j}$$
$$\int_{0}^{1} f(x) dx \approx \int_{0}^{1}\sum_{i=0}^{n} f(x_i)*\prod_{j\neq i}\frac{x-x_j}{x_i-x_j}dx=$$
$$\int_{0}^{1}f(0)\frac{x-1/3}{-1/3}*\frac{x-2/3}{-2/3}*\frac{x-1}{-1}+f(1/3)\frac{x}{1/3}*\frac{x-2/3}{-1/3}*\frac{x-1}{-2/3}+$$
$$f(2/3)\frac{x}{2/3}*\frac{x-1/3}{1/3}*\frac{x-1}{-1/3}+f(1)\frac{x}{1}*\frac{x-1/3}{2/3}*\frac{x-2/3}{1/3}dx=$$
$$\int_{0}^{1}f(0)\frac{(x-1/3)(x-2/3)(x-1)}{-2/9}+f(1/3)\frac{(x)(x-2/3)(x-1)}{2/27}+$$
$$f(2/3)\frac{(x)(x-1/3)(x-1)}{-2/27}+f(1)\frac{(x)(x-1/3)(x-2/3)}{2/9}dx=$$
$$\int_{0}^{1}f(0)\frac{(3x-1)(3x-2)(x-1)}{-2}+f(1/3)\frac{(x)(3x-2)(x-1)}{2/9}+$$
$$f(2/3)\frac{(x)(3x-1)(x-1)}{-2/9}+f(1)\frac{(x)(3x-1)(3x-2)}{2}dx=$$
$$\int_{0}^{1}f(0)\frac{(9x^2-9x+2)(x-1)}{-2}+f(1/3)\frac{(x)(3x^2-5x+2)}{2/9}+$$
$$f(2/3)\frac{(x)(3x^2-4x+1)}{-2/9}+f(1)\frac{(x)(9x^2-9x+2)}{2}dx=$$
$$\int_{0}^{1}f(0)\frac{9x^3-18x^2+11x-2}{-2}+f(1/3)\frac{3x^3-5x^2+2x}{2/9}+$$
$$f(2/3)\frac{(3x^3-4x^2+x)}{-2/9}+f(1)\frac{9x^3-9x^2+2x}{2}dx=$$
$$f(0)\frac{9/4-6+11/2-2}{-2}+f(1/3)\frac{3/4-5/3+1}{2/9}+f(2/3)\frac{3/4-4/3+1/2}{-2/9}+f(1)\frac{9/4-3+1}{2}=$$
$$\frac{1}{8}f(0)+\frac{3}{8}f(1/3)+\frac{3}{8}f(2/3)+\frac{1}{8}f(1)$$

\begin{exercise}

10.2
\end{exercise}
Find 
$$\int_0^1f(x)dx=A_0f(0)+A_1f(1)$$
witch is exact for all $f(x)=ae^x+b\cos(\pi x/2)$.
If such a pair $A_0,A_1$ exist then they must be exact for $e^x$ and $\cos(\pi x/2)$ in other words $e-1=A_0+eA_1$ and $2/\pi=A_0$.  Thus 
$$\int_0^1f(x)dx=\frac{2}{\pi} f(0)+\frac{e-1-2/\pi}{e} f(1)$$
is exact for all $f(x)=ae^x+b\cos(\pi x/2)$.












\end{document}