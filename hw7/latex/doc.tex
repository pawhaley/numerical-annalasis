\documentclass[12pt]{article}

\usepackage{amssymb,amsmath,amsthm}
\usepackage[top=1in, bottom=1in, left=1.25in, right=1.25in]{geometry}
\usepackage{fancyhdr}
\usepackage{enumerate}
\usepackage[bw,framed,numbered]{mcode}
\usepackage{graphicx}

% Comment the following line to use TeX's default font of Computer Modern.
\usepackage{times,txfonts}

\newtheoremstyle{homework}% name of the style to be used
  {18pt}% measure of space to leave above the theorem. E.g.: 3pt
  {12pt}% measure of space to leave below the theorem. E.g.: 3pt
  {}% name of font to use in the body of the theorem
  {}% measure of space to indent
  {\bfseries}% name of head font
  {:}% punctuation between head and body
  {2ex}% space after theorem head; " " = normal interword space
  {}% Manually specify head
\theoremstyle{homework} 

% Set up an Exercise environment and a Solution label.
\newtheorem*{exercisecore}{Exercise \@currentlabel}
\newenvironment{exercise}[1]
{\def\@currentlabel{#1}\exercisecore}
{\endexercisecore}

\newcommand{\localhead}[1]{\par\smallskip\noindent\textbf{#1}\nobreak\\}%
\newcommand\solution{\localhead{Solution:}}

%%%%%%%%%%%%%%%%%%%%%%%%%%%%%%%%%%%%%%%%%%%%%%%%%%%%%%%%%%%%%%%%%%%%%%%%
%
% Stuff for getting the name/document date/title across the header
\makeatletter
\RequirePackage{fancyhdr}
\pagestyle{fancy}
\fancyfoot[C]{\ifnum \value{page} > 1\relax\thepage\fi}
\fancyhead[L]{\ifx\@doclabel\@empty\else\@doclabel\fi}
\fancyhead[C]{\ifx\@docdate\@empty\else\@docdate\fi}
\fancyhead[R]{\ifx\@docauthor\@empty\else\@docauthor\fi}
\headheight 15pt

\def\doclabel#1{\gdef\@doclabel{#1}}
\doclabel{Use {\tt\textbackslash doclabel\{MY LABEL\}}.}
\def\docdate#1{\gdef\@docdate{#1}}
\docdate{Use {\tt\textbackslash docdate\{MY DATE\}}.}
\def\docauthor#1{\gdef\@docauthor{#1}}
\docauthor{Use {\tt\textbackslash docauthor\{MY NAME\}}.}
\makeatother

% Shortcuts for blackboard bold number sets (reals, integers, etc.)
\newcommand{\Reals}{\ensuremath{\mathbb R}}
\newcommand{\Nats}{\ensuremath{\mathbb N}}
\newcommand{\Ints}{\ensuremath{\mathbb Z}}
\newcommand{\Rats}{\ensuremath{\mathbb Q}}
\newcommand{\Cplx}{\ensuremath{\mathbb C}}
%% Some equivalents that some people may prefer.
\let\RR\Reals
\let\NN\Nats
\let\II\Ints
\let\CC\Cplx

%%%%%%%%%%%%%%%%%%%%%%%%%%%%%%%%%%%%%%%%%%%%%%%%%%%%%%%%%%%%%%%%%%%%%%%%%%%%%%%%%%%%%%%
%%%%%%%%%%%%%%%%%%%%%%%%%%%%%%%%%%%%%%%%%%%%%%%%%%%%%%%%%%%%%%%%%%%%%%%%%%%%%%%%%%%%%%%
% 
% The main document start here.

% The following commands set up the material that appears in the header.

%%%%%%%%%%%%%%%%%%%%%%%%%%%%%%%%%%%%%%%%%%%%%%%%%%%%%%%%%%%%%%%%%%%%%%%%%%%%%%%%%%%%%%%
%%%%%%%%%%%%%%%%%%%%%%%%%%%%%%%%%%%%%%%%%%%%%%%%%%%%%%%%%%%%%%%%%%%%%%%%%%%%%%%%%%%%%%%
% 
% The main document start here.

% The following commands set up the material that appears in the header.
\doclabel{Math 310: Homework 6}
\docauthor{Parker Whaley}
\docdate{September 9, 2016}

\newcommand{\vv}{\mathbf{v}}
\begin{document}
\begin{exercise}

Use the worksheet to wite down the $PLU$ factorization.
\end{exercise}
$$P_1=\begin{bmatrix}
1 & 0 & 0\\
0 & 1 & 0\\
0 & 0 & 1
\end{bmatrix},
L_1=\begin{bmatrix}
1 & 0 & 0\\
0 & 1 & 0\\
0 & 0 & 1
\end{bmatrix},
U_1=\begin{bmatrix}
1/3 & 0 & 0\\
1 & 1 & 1\\
1/2 & 0 & 1
\end{bmatrix}$$

$$P_2=\begin{bmatrix}
0 & 1 & 0\\
1 & 0 & 0\\
0 & 0 & 1
\end{bmatrix},
L_2=\begin{bmatrix}
1 & 0 & 0\\
0 & 1 & 0\\
0 & 0 & 1
\end{bmatrix},
U_2=\begin{bmatrix}
1 & 1 & 1\\
1/3 & 0 & 0\\
1/2 & 0 & 1
\end{bmatrix}$$

$$P_3=\begin{bmatrix}
0 & 1 & 0\\
1 & 0 & 0\\
0 & 0 & 1
\end{bmatrix},
L_3=\begin{bmatrix}
1 & 0 & 0\\
1/3 & 1 & 0\\
1/2 & 0 & 1
\end{bmatrix},
U_3=\begin{bmatrix}
1 & 1 & 1\\
0 & -1/3 & -1/3\\
0 & -1/2 & 1/2
\end{bmatrix}$$

$$P_4=\begin{bmatrix}
0 & 1 & 0\\
0 & 0 & 1\\
1 & 0 & 0
\end{bmatrix},
L_4=\begin{bmatrix}
1 & 0 & 0\\
1/2 & 1 & 0\\
1/3 & 0 & 1
\end{bmatrix},
U_4=\begin{bmatrix}
1 & 1 & 1\\
0 & -1/2 & 1/2\\
0 & -1/3 & -1/3
\end{bmatrix}$$

$$P_5=\begin{bmatrix}
0 & 1 & 0\\
0 & 0 & 1\\
1 & 0 & 0
\end{bmatrix},
L_5=\begin{bmatrix}
1 & 0 & 0\\
1/2 & 1 & 0\\
1/3 & 2/3 & 1
\end{bmatrix},
U_5=\begin{bmatrix}
1 & 1 & 1\\
0 & -1/2 & 1/2\\
0 & 0 & -2/3
\end{bmatrix}$$

\begin{exercise}

Let $$A=\begin{bmatrix}
10^{-16} & 1\\
1 & 1
\end{bmatrix},
b=\begin{bmatrix}
2\\
3
\end{bmatrix}$$
\end{exercise}
\begin{enumerate}[(a)]
\item
Solve $Ax=b$ exactly.
$$\begin{bmatrix}
10^{-16} & 1\\
1 & 1
\end{bmatrix}x=\begin{bmatrix}
2\\
3
\end{bmatrix}$$

$$\begin{bmatrix}
10^{-16} & 1\\
0 & 1-10^{16}
\end{bmatrix}x=\begin{bmatrix}
2\\
3-2*10^{16}
\end{bmatrix}$$

$$\begin{bmatrix}
10^{-16} & 0\\
0 & 1-10^{16}
\end{bmatrix}x=\begin{bmatrix}
2-\frac{3-2*10^{16}}{1-10^{16}}\\
3-2*10^{16}
\end{bmatrix}$$

$$\begin{bmatrix}
1 & 0\\
0 & 1
\end{bmatrix}x=\begin{bmatrix}
\frac{2}{10^{-16}}-\frac{3-2*10^{16}}{10^{-16}(1-10^{16})}\\
\frac{3-2*10^{16}}{1-10^{16}}
\end{bmatrix}$$

$$x=
\begin{bmatrix}
\frac{2}{10^{-16}}-\frac{3-2*10^{16}}{10^{-16}(1-10^{16})}\\
\frac{3-2*10^{16}}{1-10^{16}}
\end{bmatrix}=
\begin{bmatrix}
\frac{2-2*10^{16}-(3-2*10^{16})}{10^{-16}(1-10^{16})}\\
\frac{3-2*10^{16}}{1-10^{16}}
\end{bmatrix}=
\begin{bmatrix}
\frac{-1}{10^{-16}-1}\\
\frac{3-2*10^{16}}{1-10^{16}}
\end{bmatrix}\approx
\begin{bmatrix}
1\\
2
\end{bmatrix}
$$
As a check:
$$Ax=\begin{bmatrix}
10^{-16} & 1\\
1 & 1
\end{bmatrix}
\begin{bmatrix}
1\\
2
\end{bmatrix}=
\begin{bmatrix}
10^{-16}+2\\
3
\end{bmatrix}\approx
\begin{bmatrix}
2\\
3
\end{bmatrix}
$$
\item
What is the 2-norm condition number for $A$?  Is A well behaved in the 2-norm?\\
The condition number is $2.6180$.  This is well conditioned since errors is output will be on the same order as errors in imput.
\item
Here is my solution to solving a matrix without pivoting
\lstinputlisting{../octave/matrixSolve.m}
And the solution I get:
\lstinputlisting{../octave/d1.txt}
The problem with this method is that adding and subtracting large numbers causes errors:
\lstinputlisting{../octave/d2.txt}
\end{enumerate}
\begin{exercise}

From the worksheet on implementing partial pivoting, show your code for mylu.m. Then
show your answer to problem 10. Also, use your function usolve from the last homework
and lsolve from the course web page to solve $Ax = b$ where $A$ is the matrix from
problem 10 of the worksheet and $b = \begin{bmatrix}-1\\ 6\\ -8\end{bmatrix}$.
\end{exercise}
\lstinputlisting{../octave/mylu.m}
This is my demonstration of L U factorization:
\lstinputlisting{../octave/d3.txt}
Here are my usolve, isolve, and general solve without partial pivoting:
\lstinputlisting{../octave/usolve.m}
\lstinputlisting{../octave/lsolve.m}
\lstinputlisting{../octave/noPPsolve.m}
So solving the desired problem:
\lstinputlisting{../octave/d4.txt}

\begin{exercise}

To compute the inverse of an $n \times n$ matrix $A$ you need to find n vectors $v_i$ such that
$Av_i = e_i$
, where $e_i$
is the vector of all zeros, except that $e_i$ has a one in its $i$
th
entry. E.g., if $n = 4$ then $e_3 = [ 0, 0,1, 0]^T$. Once the vectors $v_i$ are known, then
$A^{-1} = [v_1 | v_2 | \cdots |v_n]$. 
How many floating point operations are required to perform an LU decomposition of A
and then solve for the n vectors $v_i$?
If one wants to compute the solution of $Ax = b$ by computing $A^{-1}$ and then mutiplying to
obtain $x = A^{-1}b$, how many floating point operations does this take? Compare this number
with the number of floating point operations to solve $Ax = b$ by LU decomposition
without computing $A^{-1}$
.
\end{exercise}
Doing a LU decomposition, lets calculate the cost by examining the solution, converting for loops into summations, we see imeadiately that there are $$\sum^{n-1}_1\sum^n_{i+1}1=\sum^{n-1}_1n-i=(n-1)n-\frac{(n-1)n}{2}=1/2n^2-1/2n$$ devisions.  There will be the same number of subtractions as multiplications so we get  $$\sum^{n-1}_1\sum^n_{i+1}\sum^n_{i+1}1=\sum^{n-1}_1\sum^n_{i+1}n-i=\sum^{n-1}_1 (n-i)(n-i)=\sum^{n-1}_1 n^2-2ni+i^2=n^2(n-1)-2n\frac{(n-1)n}{2}+\frac{(n-1)(n)(2n-1)}{6}$$
multiplications or devisions.  Well these numbers are nasty so lets work with order of magnitude.  LU factorization costs $\frac{2n^3}{3}+O(n^2)$ FLOPs.\\
Solving a unit lower triangular matrix would require a multiply and a add for each entry below the diagonal, this would be $n^2/2-n$ adds and multiplys.  The solve on the upper triangular matrix works out the same but there are n additional devisions, since the diagonal is not zeroed.  so a compleate LU solve requires $4*(n^2/2-n)+n=2n^2+O(n)$ FLOPs.\\
If we want to find the inverse of a matrix by doing LU factoring and n LU solves it will take $\frac{2n^3}{3}+O(n^2)+n(2n^2+O(n))=2\frac{2n^3}{3}+O(n^2)$, If we then multiply to solve we take a additional $2n^2+O(n)$ operations for a total of $2\frac{2n^3}{3}+O(n^2)$ FLOPs.  This is much worse than a simple LU factor and solve method since a LU factor and solve method only takes $\frac{2n^3}{3}+O(n^2)+(2n^2+O(n))=\frac{2n^3}{3}+O(n^2)$ FLOPs.

\begin{exercise}
worksheet problem 16
\end{exercise}
\lstinputlisting{../octave/d5.txt}
\begin{exercise}
worksheet problem 17
\end{exercise}
This is a working solution
\lstinputlisting{../octave/myplu.m}
Here is a demonstration that it works:
\lstinputlisting{../octave/d6.txt}
\end{document}