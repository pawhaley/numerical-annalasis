\documentclass[12pt]{article}

\usepackage{amssymb,amsmath,amsthm}
\usepackage[top=1in, bottom=1in, left=1.25in, right=1.25in]{geometry}
\usepackage{fancyhdr}
\usepackage{enumerate}
\usepackage[bw,framed,numbered]{mcode}
\usepackage{graphicx}

% Comment the following line to use TeX's default font of Computer Modern.
\usepackage{times,txfonts}

\newtheoremstyle{homework}% name of the style to be used
  {18pt}% measure of space to leave above the theorem. E.g.: 3pt
  {12pt}% measure of space to leave below the theorem. E.g.: 3pt
  {}% name of font to use in the body of the theorem
  {}% measure of space to indent
  {\bfseries}% name of head font
  {:}% punctuation between head and body
  {2ex}% space after theorem head; " " = normal interword space
  {}% Manually specify head
\theoremstyle{homework} 

% Set up an Exercise environment and a Solution label.
\newtheorem*{exercisecore}{Exercise \@currentlabel}
\newenvironment{exercise}[1]
{\def\@currentlabel{#1}\exercisecore}
{\endexercisecore}

\newcommand{\localhead}[1]{\par\smallskip\noindent\textbf{#1}\nobreak\\}%
\newcommand\solution{\localhead{Solution:}}

%%%%%%%%%%%%%%%%%%%%%%%%%%%%%%%%%%%%%%%%%%%%%%%%%%%%%%%%%%%%%%%%%%%%%%%%
%
% Stuff for getting the name/document date/title across the header
\makeatletter
\RequirePackage{fancyhdr}
\pagestyle{fancy}
\fancyfoot[C]{\ifnum \value{page} > 1\relax\thepage\fi}
\fancyhead[L]{\ifx\@doclabel\@empty\else\@doclabel\fi}
\fancyhead[C]{\ifx\@docdate\@empty\else\@docdate\fi}
\fancyhead[R]{\ifx\@docauthor\@empty\else\@docauthor\fi}
\headheight 15pt

\def\doclabel#1{\gdef\@doclabel{#1}}
\doclabel{Use {\tt\textbackslash doclabel\{MY LABEL\}}.}
\def\docdate#1{\gdef\@docdate{#1}}
\docdate{Use {\tt\textbackslash docdate\{MY DATE\}}.}
\def\docauthor#1{\gdef\@docauthor{#1}}
\docauthor{Use {\tt\textbackslash docauthor\{MY NAME\}}.}
\makeatother

% Shortcuts for blackboard bold number sets (reals, integers, etc.)
\newcommand{\Reals}{\ensuremath{\mathbb R}}
\newcommand{\Nats}{\ensuremath{\mathbb N}}
\newcommand{\Ints}{\ensuremath{\mathbb Z}}
\newcommand{\Rats}{\ensuremath{\mathbb Q}}
\newcommand{\Cplx}{\ensuremath{\mathbb C}}
%% Some equivalents that some people may prefer.
\let\RR\Reals
\let\NN\Nats
\let\II\Ints
\let\CC\Cplx

%%%%%%%%%%%%%%%%%%%%%%%%%%%%%%%%%%%%%%%%%%%%%%%%%%%%%%%%%%%%%%%%%%%%%%%%%%%%%%%%%%%%%%%
%%%%%%%%%%%%%%%%%%%%%%%%%%%%%%%%%%%%%%%%%%%%%%%%%%%%%%%%%%%%%%%%%%%%%%%%%%%%%%%%%%%%%%%
% 
% The main document start here.

% The following commands set up the material that appears in the header.

%%%%%%%%%%%%%%%%%%%%%%%%%%%%%%%%%%%%%%%%%%%%%%%%%%%%%%%%%%%%%%%%%%%%%%%%%%%%%%%%%%%%%%%
%%%%%%%%%%%%%%%%%%%%%%%%%%%%%%%%%%%%%%%%%%%%%%%%%%%%%%%%%%%%%%%%%%%%%%%%%%%%%%%%%%%%%%%
% 
% The main document start here.

% The following commands set up the material that appears in the header.
\doclabel{Math 310: Homework 3}
\docauthor{Parker Whaley}
\docdate{August 22, 2016}

\newcommand{\vv}{\mathbf{v}}
\begin{document}
\begin{exercise}

Write down the 4th order Taylor polynomial of $\sqrt{
x}$ centered at $x = 1$. Let $P(x)$ denote
this polynomial. If $1 \leq x \leq 2$, what can you say about the size of $|\sqrt{x} - P(x)|$? Hint: Use
the remainder term!
\end{exercise}
Lets start by geting the derivitives of $f(x)=\sqrt{x}$.
$$f(x)=x^\frac{1}{2}$$
$$f'(x)=\frac{1}{2}x^{-\frac{1}{2}}$$
$$f''(x)=-\frac{1}{4}x^{-\frac{3}{2}}$$
$$f'''(x)=\frac{3}{8}x^{-\frac{5}{2}}$$
$$f^{(4)}(x)=-\frac{15}{16}x^{-\frac{7}{2}}$$
$$f^{(5)}(x)=\frac{105}{32}x^{-\frac{9}{2}}$$

Now we can construct $P(x)$ as 
$$P(x)=1^\frac{1}{2}+\frac{1}{2}1^{-\frac{1}{2}}(x-1)+-\frac{1}{2}\frac{1}{4}1^{-\frac{3}{2}}(x-1)^2+\frac{1}{6}\frac{3}{8}1^{-\frac{5}{2}}(x-1)^3+-\frac{1}{24}\frac{15}{16}1^{-\frac{7}{2}}(x-1)^4$$
$$P(x)=1+\frac{1}{2}(x-1)-\frac{1}{8}(x-1)^2+\frac{1}{16}(x-1)^3-\frac{5}{128}(x-1)^4$$
Note that the remander term would be
$$R(x)=\frac{1}{5!}\frac{105}{32}\xi^{-\frac{9}{2}}(x-1)^5=\frac{7}{256}\xi^{-\frac{9}{2}}(x-1)^5$$
Where $f(x)=P(x)+R(x)$.
We are now asked to compute the error, $E$, in the case that $1 \leq x \leq 2$. We see that $E=|f(x) - P(x)|=|R(x)|=|\frac{7}{256}\xi^{-\frac{9}{2}}(x-1)^5|$ and noting that $\xi,x\in [1,2]$ we get $E=\frac{7}{256}\xi^{-\frac{9}{2}}(x-1)^5$.  The maximum value E could have would be $x=2$ and $\xi=1$ so $E_{max}=\frac{7}{256}$ and of course the smallest value is at $x=1$ where by our construction $f(x)=p(x)$ so $0\leq E\leq \frac{7}{256}$.
\begin{exercise}

Chapter 4: 2 (b)
\end{exercise}
\lstinputlisting{../octave/newton.m}
\end{document}